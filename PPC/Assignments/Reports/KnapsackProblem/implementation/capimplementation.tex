\chapter{Implementation} \label{cap:implementation}


In this assignment, three distinct parallelization strategies were implemented:

\begin{enumerate}
    \item \textbf{Master-Worker Pattern with Poison Pill}
    \item \textbf{Scatter-Gather Pattern}
    \item \textbf{Fork-Join Pattern}
\end{enumerate}

All these strategies apply parallelization to each generation step of the
algorithm, specifically targeting the methods that evaluate the fitness of
individuals in the population. This focus is strategic since fitness evaluation
is the most computationally intensive task and benefits significantly from
parallel execution.

\section{Master-Worker Pattern with Poison Pill} \label{sec:master-worker}

The Master-Worker uses a master thread to be responsible for distributing tasks
to multiple worker threads. This implementation creates a thread pool with a
configurable number of workers (\texttt{maxThreads} parameter, defaulting to
\texttt{Runtime.getRuntime().availableProcessors()}). The master divides the
population into equal chunks based on a calculated chunk size (\texttt{POP\_SIZE
/ maxThreads}), ensuring balanced workload distribution among workers. \\
\\
The implementation consists of three key components:

\subsection{Core Classes}

\begin{itemize}
    \item \texttt{Worker} class that implements \texttt{Runnable} - represents a
    worker thread that continuously polls the task queue for work
    \item \texttt{Task} class - encapsulates the work to be performed,
    containing a \texttt{TaskType} and a \texttt{Runnable} with the actual logic
    \item \texttt{TaskType} enum - distinguishes between regular tasks
    (\texttt{RUNNABLE}) and termination signals (\texttt{POISON\_PILL})
\end{itemize}

\subsection{Architecture}
The master maintains a thread-safe \texttt{LinkedBlockingQueue<Task>} that
serves as the communication channel between the master and workers. Workers are
initialized once at the beginning of the algorithm and remain active throughout
all generations, continuously processing tasks as they become available.

\subsection{Parallelized Operations}
This pattern is applied to four critical algorithm operations:

\begin{enumerate}
    \item \textbf{Fitness Calculation} (\texttt{calculateFitness()}): Each
    worker evaluates the fitness of individuals in its assigned chunk of the
    population
    \item \textbf{Best Individual Selection} (\texttt{bestOfPopulation()}):
    Workers find the best individual in their chunk using
    \texttt{AtomicReference} with compare-and-swap operations to safely update
    the global best
    \item \textbf{Population Reproduction} (\texttt{calculateBestPopulation()}):
    Workers perform tournament selection and crossover operations to generate
    new individuals
    \item \textbf{Mutation} (\texttt{mutate()}): Workers apply mutation to
    individuals in their assigned range based on the mutation probability
\end{enumerate}

\subsection{Synchronization}
Each parallelized step utilizes a \texttt{CountDownLatch} initialized with the
number of workers. As each worker completes its assigned chunk of work, it calls
\texttt{countDown()}, while the master waits using \texttt{await()} before
proceeding to the next step. This mechanism ensures proper synchronization
between generations.

\subsection{Lifecycle Management}
Workers are initialized once at the beginning (\texttt{startWorkers()}) and
terminated at the end by enqueueing a poison pill for each worker
(\texttt{stopWorkers()}). This approach minimizes thread creation overhead and
maintains consistent performance across generations.

\newpage

\section{Scatter-Gather Pattern}\label{sec:scatter-gather}

The Scatter-Gather pattern divides the population into smaller chunks
(scattered) across multiple threads for parallel processing, then collects the
results (gathered) after completion. This implementation utilizes Java's
\texttt{ExecutorService} with a fixed thread pool to manage parallel execution
efficiently.

\subsection{Core Architecture}

\begin{itemize}
    \item Uses \texttt{Executors.newFixedThreadPool(maxThreads)} where
    \texttt{maxThreads} is a configurable parameter (defaulting to
    \texttt{Runtime.getRuntime().availableProcessors()})
    \item Employs a centralized \texttt{computeFutures()} method that handles
    the scatter-gather logic for all parallelized operations
    \item Creates \texttt{Future} objects to track and synchronize the
    completion of parallel tasks
\end{itemize}

\subsection{Scatter Phase}
The \texttt{computeFutures()} method divides the work range into equal chunks
based on \texttt{chunkSize = total operations / maxThreads}. Each thread
receives a specific range to process:

\begin{itemize}
    \item Thread 0: processes indices \texttt{[start, start + chunkSize)}
    \item Thread 1: processes indices \texttt{[start + chunkSize, start + 2 *
    chunkSize)}
    \item Last thread: processes remaining indices to handle any remainder from
    integer division
\end{itemize}

\subsection{Gather Phase}

After submitting all tasks to the thread pool, the method calls
\texttt{futures.forEach \{ it.get() \}} to wait for all threads to complete
their work before proceeding. This ensures synchronization between parallel
operations.

\subsection{Parallelized Operations}\label{sec:scatter-gather-operations}

\begin{enumerate}
    \item \textbf{Fitness Calculation} (\texttt{calculateFitness()}): Each
    thread evaluates fitness for individuals in its assigned range
    \item \textbf{Best Individual Selection} (\texttt{bestOfPopulation()}):
    Threads find local best individuals and use \texttt{AtomicReference} with
    compare-and-swap to update the global best
    \item \textbf{Population Reproduction} (\texttt{calculateBestPopulation()}):
    Each thread performs tournament selection and crossover for its assigned
    range
    \item \textbf{Mutation} (\texttt{mutate()}): Threads apply mutation to
    individuals in their assigned range based on mutation probability
\end{enumerate}

\subsection{Thread Management}
The thread pool is created once per algorithm execution and reused across all
generations, minimizing thread creation overhead. The pool is properly shut down
in a \texttt{finally} block to ensure resource cleanup.

\section{Fork-Join Pattern}

The Fork-Join pattern utilizes Java's Fork-Join framework to recursively divide
tasks into smaller subtasks until they reach a manageable size (threshold), then
processes them in parallel and combines the results. This implementation
leverages \texttt{ForkJoinPool} and \texttt{RecursiveAction} to achieve
work-stealing parallelism.

\subsection{Core Architecture}

\begin{itemize}
    \item Utilizes \texttt{ForkJoinPool(maxThreads)} where \texttt{maxThreads}
    is a configurable parameter (defaulting to
    \texttt{Runtime.getRuntime().availableProcessors()})
    \item Implements \texttt{RecursiveAction} for each algorithm operation
    requiring parallelization
    \item Employs a configurable threshold (default 1000) to determine when to
    stop subdividing tasks
    \item Uses \texttt{invokeAll()} to fork subtasks and automatically join
    their results
\end{itemize}

\subsection{Fork Phase (Task Subdivision)}
The \texttt{computeRange()} method implements the recursive subdivision logic:

\begin{itemize}
    \item If the work range \texttt{(end - start)} is smaller than or equal to
    the threshold, executes the task directly
    \item Otherwise, splits the range in half at the midpoint: \texttt{mid =
    (start + end) / 2}
    \item Creates two \texttt{RecursiveAction} subtasks: left half
    \texttt{[start, mid)} and right half \texttt{[mid, end)}
    \item Uses \texttt{invokeAll(left, right)} to fork both subtasks for
    parallel execution
\end{itemize}

\newpage

\subsection{Join Phase (Result Combination)}
The Fork-Join framework automatically handles the join phase through the
\texttt{invokeAll()} method, which:

\begin{itemize}
    \item Executes subtasks in parallel using available worker threads
    \item Implements work-stealing where idle threads can steal work from busy
    threads' queues
    \item Waits for all subtasks to complete before returning control to the
    caller
\end{itemize}

\subsection{Parallelized Operations}

\begin{enumerate}
    \item \textbf{Fitness Calculation} (\texttt{calculateFitness()}):
    Recursively divides the population range and evaluates fitness for
    individuals in parallel
    \item \textbf{Best Individual Selection} (\texttt{bestOfPopulation()}):
    Finds local best individuals in parallel chunks and uses
    \texttt{AtomicReference} with compare-and-swap to update the global best
    \item \textbf{Population Reproduction} (\texttt{calculateBestPopulation()}):
    Performs tournament selection and crossover operations in parallel across
    population segments
    \item \textbf{Mutation} (\texttt{mutate()}): Applies mutation to individuals
    in parallel based on the mutation probability
\end{enumerate}

\subsection{Work-Stealing Benefits}
The Fork-Join framework provides automatic load balancing through work-stealing,
where threads that complete their work early can steal tasks from other threads'
work queues, maximizing CPU utilization and minimizing idle time.

\subsection{Threshold Optimization}
The configurable threshold parameter allows fine-tuning the granularity of
parallelization - smaller thresholds create more parallelism but increase
overhead, while larger thresholds reduce overhead but may limit parallelization
opportunities.